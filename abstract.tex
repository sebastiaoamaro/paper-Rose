\begin{abstract}
Distributed systems form the backbone of critical infrastructures, yet remain vulnerable to \emph{external-fault-induced bugs} that manifest only when specific external events occur during specific application states. 
Existing approaches to reproducing these bugs require fine-grained information about the application, which might not be available in production, and operate within a limited fault model.
\sys is a novel black-box approach that collects traces from production environments and systematically generates fault schedules that reproduce these bugs.
By leveraging the insight that external faults are observable through system interfaces, \sys uses lightweight tracing (2.6\% overhead) to capture essential system-environment interactions.
Then, it identifies the application states when faults must occur to trigger bugs, and generates schedules that consistently reproduce these bugs.
\sys successfully reproduced 20 bugs across eight production systems implemented in diverse languages (C, C++, Java, Go, Scala), including widely-used systems such as Zookeeper, MongoDB, and HBase.
\end{abstract}
