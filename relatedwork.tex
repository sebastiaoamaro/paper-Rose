\section{Related Work}

In this section, we analyze existing approaches that use fault injection techniques for bug finding and reproduction.

\mypara{Bug Finding Using Fault Injection.} 
Fault injection systems fall into two categories: general-purpose frameworks and specialized tools.
General frameworks like Jepsen~\cite{jepsen}  employ randomized fault injection to test invariants across diverse systems, while Mallory~\cite{mallory} extends Jepsen with  reinforcement learning techniques that rewards fault sequences revealing new system behaviors.
FaultSee~\cite{9237004} proposes a domain-specific language to specify fault injection scenarios, and Chaos Monkey~\cite{chaosmonkey} focuses primarily on process termination within production environments.

More specialized tools target specific system categories or fault types.
PACE~\cite{AlagappanEtAl16-OSDI} explores correlated crash vulnerabilities in distributed filesystems by systematically generating persistent states, while Sieve~\cite{sieve}  targets testing the reliability of cluster management controllers.
FCatch~\cite{fcatch} aims to find timing-sensitive bugs in cloud systems, specifically Time-of-Fault bugs, while CrashMonkey~\cite{crashmonkey18mohan} focuses on uncovering   crash consistency issues in file systems.
Legolas~\cite{legolas} uses static analysis to extract abstract states from system code with the goal of exposing partil failure bugs in distributed systems.
Several Byzantine fault tolerance testing frameworks~\cite{toolbft, bftcc, twins, byzzfuzz, tyr} leverage the internal system state to identify critical fault injection points.

These approaches leverage various techniques to efficiently a vast fault space to discover potential bugs.
However, their goals are fundamentally different than \sys's which focuses on precisely reproducing known bugs with minimal overhead.
Furthermore, most of these approaches require extensive instrumentation and/or leverage application-level knowledge and hence operate at higher abstraction level than \sys OS-level approach.


\mypara{Bug Reproduction.}
Record-and-replay systems~\cite{r2,ireplayer,doubleplay,mozilarr} capture all non-deterministic behavior, including program inputs and thread interleavings, and later replay this exact sequence of events.
While comprehensive, these approaches impose prohibitive overhead (e.g., RR~\cite{mozilarr} has a minimum overhead of 1.49×), making them impractical for fault reproduction in production systems.


ReproLite~\cite{reprolite} proposes a domain-specific language for expressing bug scenarios (similar to our fault schedule) and enforces the execution order specified in the DSL.
It does not specifically search for the necessary  context at which faults must be injected but rather serves as a support tool to validate developer's conjectures about faulty scenarios.
Pensive~\cite{pensieve} processes log files and system bytecode to automatically reconstruct the sequence of inputs (API calls) that trigger a given bug.
The approach is specific to Java-based systems and focus on uncovering faulty inputs that trigger bugs rather than bugs triggered by external faults.
Anduril~\cite{anduril} aims to reproduce fault-induced failures and is the closest work to ours.
It extends Pensieve's algorithm to include exception handling logic and rather than searching for sequences of inputs, searches for exceptions that might lead to failures.

These systems share the common limitation that they leverage JVM-specific information from application monitoring and fault injection (in the case of Anduril) and hence cannot be broadly applied to non-JVM applications. 
\sys employs OS-level observation and fault injection techniques and hence can target systems implemented in any programming language.
The low overhead of the tracer allows it to run alongside the production system and capture the minimal amount of information necessary to reproduce the bug.
Finally \sys's progressive fault context refinement efficiently identifies minimal reproduction conditions to trigger the bug with high replay rates (>90\% in most cases).



