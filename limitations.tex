\section{Limitations and Future Work}
We now discuss some limitations of \sys and how these can be addressed in future work.

\mypara{False negatives.}
The current implementation might have false negatives during fault schedule evaluation.
When a schedule fails to trigger the bug during the initial testing, \sys discards it entirely, even though subsequent executions might have achieved a statistically significant replay rate.
A possible enhancement would involve executing each candidate schedule multiple times to establish statistical confidence, though this would linearly increase reproduction time.


\mypara{Concurrency}
The current implementation also lacks support for thread-specific fault contexts.
While this information can be captured by the tracer, our fault context refinement does not consider thread-specific conditions.
Even though we did not encounter bugs that required this thread-specific context, we plan to add this functionality in future work.


\mypara{Unsupported operations}
A more fundamental constraint of \sys's design is the ability to handle faults in memory-mapped I/O operations.
Since most read and write accesses to memory-mapped files bypass system calls, this creates a blind spot in \sys observability model where faults occurring within memory-mapped regions cannot be detected through system call monitoring alone.
Therefore, bugs triggered by memory access errors in memory-mapped regions remain outside \sys current reproduction capabilities.
Addressing this limitation in a consistent manner requires further research.

\mypara{Tracing functions in JVM}
\sa{New paragraph, R1C6:
	Our approach to detect application functions relies on attaching uprobes to binary addresses. However, this is not a possible approach for Java systems, since they run on the JVM.
	Other approaches to implement this task exist such as the use of User Statically-Defined Tracing~\cite{usdt}, or annotations with AspectJ\cite{aspectj} which can be done automatically since we know the name of the methods.
	We consider this to be implementation work, which we intend to add in future work.}
