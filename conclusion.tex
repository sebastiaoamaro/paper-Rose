\section{Conclusion}
Reproducing \efibshort is challenging since they are only exposed when external faults happen at specific points in the execution.
Existing state-of-the-art tools that target this problem are JVM-specific and emulate faults by throwing exceptions at strategic points in the code. 
This limits their applicability outside of the JVM ecosystem and fails to capture external faults that do not necessarily raise exceptions such as network delays.

This paper introduces \sys, a novel approach for reproducing \efib by relying only on OS-level monitoring and fault injection mechanisms.
%\sys addresses a fundamental gap in systems debugging: while numerous techniques exist for bug discovery, the reproduction of known bugs  --- particularly those triggered by external events --- remains challenging and time consuming.
%Our approach provides several key advantages over existing reproduction techniques.
\sys is language-agnostic, enabling the reproduction of bugs across diverse programming environments --- our evaluation reproduces 20 bugs across eight widely-used distributed systems implemented various languages, namely HBase (Java), HDFS (Java), Kafka (Java/Scala), MongoDB (C++), RedisRaft (C), Redpanda (C++), Tendermint (Go),  and Zookeeper (Java).
Moreover, \sys tracer is able to capture sufficient information for reliable bug reproduction with a low overhead of 2.6\% per node.
This allows deployment alongside production systems where performance constraints preclude extensive monitoring, and hence enables capturing information about bugs that happen seldom.
Finally, by employing a Diagnosis phase where we progressively refine the context where faults should be injected, \sys builds schedules that reproduce \efibshort with high replay rates (>90\%).
These characteristics allow developers to be more efficient in reproducing \efib and, we argue, contribute to improving distributed systems' reliability.

%Reproducing a fault-induced bug is a key step in the process of debugging \efib. Current approaches only work for a subset of systems. To help developers reproduce bugs in other systems, we developed \sys a black-box approach to reproduce \efib automatically. With \sys we can trace production systems efficiently (less than 2.5\% overhead), find the external events and the key context at which these should occur. This is accomplished by progressively refining the context while maintaining the behavior observed in production. 
%We achieve this by iteratively creating fault-schedules. \sys runs these schedules in a testing environment and leverages the feedback from each fault schedule to decide the context of the faults for the next schedule. 
%With this approach, we automatically reproduced 20 bugs on different systems written in different languages.
